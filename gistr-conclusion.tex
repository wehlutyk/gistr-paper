\section{Discussion}\label{sec:gistr-discussion}

We set out to better understand the process at work in the short term
evolution of linguistic content. In an approach complementary to the
previous chapter, we decided to design a controlled experiment that
would provide the complete data needed to model the process. We
developed an online platform for the purpose, and after adjusting task
difficulty and source complexity we were able to gather relatively large
data sets of linguistic transmission chains with low levels of spam.
Then, by combining standard NLP methods with our extension of a
biological sequence alignment algorithm, we decomposed the utterance
transformation process into small, analysable operations that subjects
use in their reformulations.

A few important points are worth noting to qualify the results we just
presented. First, several choices in the alignment procedure we followed
are sub-optimal, and were made in the interest of concrete progress. The
Needleman-Wunsch algorithm we used does not allow the detection of chunk
replacements. For instance, abbreviations or short paraphrases, which
\textcite{lauf_analyzing_2013} identify as non-negligible in their data
set (e.g. \enquote{has no idea} \(\rightarrow\) \enquote{doesn't know}),
are not captured by our approach. Extending the algorithm to allow this
is technically possible, but involves a substantial amount of additional
work, and we opted to leave such extensions for further research. The
algorithm is also blind to syntactic boundaries such as punctuation,
which insertions and deletions are likely to respect at least part of
the time. Manual inspection of the alignments showed a few cases where a
deletion would affect two contiguous parts of an utterance separated by
a comma, for which distinguishing the parts could help in improving the
final alignment. Finally, deep alignments are only explored on the basis
of optimal shallow alignments, which are not guaranteed to be the best
starting point: it is possible that the search will find a locally
optimal alignment, when a better solution could have been found by
starting from other (non-optimal) shallow alignments. Many other
improvements could be made in the future, for instance matching
insertion-deletion blocks from exchanges at different depths to overcome
local optima, or starting with local instead of global shallow
alignments. The manual evaluation of alignments indicated that the
current approach was good enough however, and that these optimisations
could indeed be left for further research without jeopardising our
model.

Concerning the design of the task, we note that the incentive provided
for subjects to perform well also leaves room for improvement (a point
related to the spam levels). Subjects had a monetary incentive with
bonuses for outstanding performance, but they did not experience the
bonus until after the experiment was completed. More generally, the
setup puts participants in the position of a subject and not of a user:
setting aside what the interface encourages, there is no intrinsic
incentive for people to put extra effort into the task. As
\textcite{gauld_experiments_1967} phrased it already long ago:
\enquote{Errors could, it seemed, be avoided, if the subject was so
inclined.} However, analysing the transformation rates of individual
subjects showed nothing to that effect: although some subjects average
better than others, it seems that no individual subject is uniformly
good or bad at the task. Initial explorations of the word span of
subjects (in Experiment 1) also showed little to no correlation to
subject performance. The best solution to this problem would be to
create an intrinsic motivation for the participants by aligning their
interests with performance. \textcite{claidiere_argumentation_2017}
implement this kind of incentive, by asking participants to defend and
convince others of a choice they have previously made. In our case, such
an incentive could guarantee subjects' involvement but would not suffice
to improve the quality of the written productions, as people will very
easily understand badly edited text in the course of a live interaction
(this would make the computational analysis even harder).

Finally, our setup entirely obviates the question of the context in
which utterances are processed and reproduced. This was a deliberate
choice, as we decided to use the simplest transmission chain task
possible and reserve the introduction of more complexity (such as
context effects) for later research. As a consequence however, we have
no control over the situation in which subjects read an utterance, nor
did we add contextual paragraphs or preceding utterances to examine
framing or priming effects on the interpretations and transformations
made by subjects. Preceding text is very likely to have effects on the
transformations, as these are reliably observed in the study of
intrusions in recall of word lists \autocite{zaromb_temporal_2006}.
Manual exploration of the data also showed (rare) cases of words from
one utterance bleeding into later transformations: in one case, a
subject reintroduced a word that they had read in an utterance three
trials earlier. This phenomenon is difficult to control beyond the
randomisation we applied, and we note that the cases we observed were
extremely rare.

In spite of these caveats, we showed that transformations can be
usefully analysed as made of bursty deletions and insertions, speckled
with word replacements (and exchanges, which we left aside in the
analysis). Deletions are by far the most frequent operation, and they
act as a gate for insertions; in turn, the size of insertions tends to
correlate to the size of deletions that they closely follow. We further
observed that all operations are less probable at the beginning of an
utterance, as well as in shorter utterances, and that deletions tend to
grow in chunk size as well as in chunk numbers towards the end of an
utterance. Finally, we observed that transformations are also bursty at
the level of the branch, suggesting that the process follows a
punctuated equilibria pattern: when a subject makes a transformation on
a previously stable utterance, the next subjects in the chain might add
transformations until the changes are regularised into a newly stable
utterance.

Overall, we suggest that adopting this descriptive model provides a
clearer picture of the process at work in the evolution of linguistic
content than has been previously achieved. The model creates an
intermediary scale between the detailed level of lexical word features
and the high level of contrasts in aggregated evolution, thus rendering
the process more intelligible. In particular, we believe that
visualisations such as the lineage plots we presented are extremely
helpful in identifying the underlying mechanisms that can be then
connected to known effects in linguistics. In the context of Cultural
Attraction Theory, this type of approach could prove useful to construct
more parsimonious models of the evolution of representations. In
particular, it brings detail to the linguistic instantiation of the
wear-and-tear and flop problems introduced by \textcite{morin_how_2016}:
the first could correspond to the way utterances are gradually
transformed by replacements, exchanges, and insertions making up for
deletions; the second could correspond to downright mass deletions in a
transformation.

Transposing the analysis from the previous chapter to the current data
set also confirmed the trends in lexical features observed in blogspace:
less frequent, longer words that are learned late and have higher
clustering coefficient are on average replaced by higher-frequency,
shorter words, learned earlier and with lower clustering coefficient. As
we discussed in the previous chapter, these words are overall easier to
produce in standard naming or word recall tasks. Examining the evolution
of these features along the branches also showed that the process
significantly transforms utterances to use easier words on average:
transformations can thus be seen as creating a gradual drift of
utterances at the low-level of lexical features due to a cognitive bias
in favour of certain word types.

While one could consider this phenomenon as relevant to Cultural
Attraction Theory, it remains extremely low-level and does not indicate
any consistent drift or attraction in the semantics of the utterances
(nor does it invalidate it). Manual inspection of the data also gave no
sign of an attraction phenomenon at the semantic level. Two points might
be noted nonetheless. First, the details of transformations seem to
follow the patterns identified by \textcite{lauf_analyzing_2013} in news
stories: complements, adverbs, modals, and more generally any details
not essential to the main meaning seem often deleted or replaced.
Second, examining episodes of bursty changes also suggested that there
is a chaotic aspect to chains: relatively minor changes in the middle of
a transformation sometimes lead to a comparatively larger change in
meaning down the branch, as an ambiguity is created and resolved
differently than previously. In Experiment 3 for instance, a subject
presented with the following utterance,

\begin{nquote} % <!-- #545 -->
  "A dozen hawkers who had been announcing the news of a non existent bomber in Kings Cross have been arrested."
\end{nquote}

introduced several changes including a typographical error that replaced
the word \enquote{news} with \enquote{new}; he or she produced the
following utterance:

\begin{nquote} % <!-- #907 -->
  "A dozen hawkers announcing non existent \textbf{new} about a bomber at Kings Cross have been arrested"
\end{nquote}

The \enquote{news} \(\rightarrow\) \enquote{new} change, while
superficially minor compared to the rest of the transformation, is quite
important once we remove the context provided by the previous utterance.
Indeed, it later went through a long regularisation such that the final
utterance of the branch read:

\begin{nquote} % <!-- #1847 -->
  "A dozen hawkers, at New Kings Cross have been arrested."
\end{nquote}

Examining the transformations in the branch suggested that the small
typographical error rendered its surroundings (\enquote{non existent
\ldots{} about a bomber}) confusing and irrelevant, such that
\enquote{new} was finally integrated as part of the \enquote{Kings
Cross} proper noun instead. This behaviour is not frequent, as many
times typographical errors are corrected by subsequent transformations,
but it appears to be possible whenever an ambiguity is created or
enhanced (not only through typographical errors).

Other intriguing semantic effects were observed. In one case for
instance, small changes that accumulated in different parts of an
utterance ended up combining into a larger semantic change, because the
relationship between parts of the utterance had eventually changed. More
broadly, tackling the question of semantic attraction (or more simply
semantic transformations) could require the definition of semantic
levels at which the transformations should be examined. As the chaotic
behaviour described above illustrates, changes at the semantic level can
be created by any ambiguity that is picked up by subjects. This can
range from a typographical error, to a change in punctuation, or a minor
change of vocabulary which seems ambiguous to one subject but not to
another. The question is thus shifted from the structure of utterances
themselves to what subjects attend to, in an utterance, for a given task
or in a given interaction situation. In other words, analysing the
semantic changes of utterances goes hand in hand with a determination of
what aspects of an utterance are relevant for a given task or
interaction, that is, it requires an approach to utterance pragmatics.

Without delving into utterance pragmatics, a more palatable development
of this approach would be to focus on a smaller number of root
utterances with less branches in each tree, and explore evolution of
content at the very long term. While such an approach would be more
dependent on the initial utterances, it would allow us to explore
higher-level evolutionary dynamics along the chain (e.g.~recurring sets
of changes). Testing for the existence of attractors, in particular,
might require such long-lived chains, as the number of iterations is an
important factor to consider. A second approach, which could be combined
with long-lived chains, would be to ask about the role of simple
semantics and syntax in the transformations: beyond lexical features and
word categorisations, one could attempt to quantify and thus
characterise the change in meaning at each transformation and overall
through the chains, possibly through deeper integration with existing
NLP methods. Semantic parsing methods, in particular, can provide a
first useful model of the semantics encoded in a sentence. Depending on
the way such information is represented, an alignment technique similar
to the one developed here could be used to study the changes in semantic
parses along a chain, thus opening the study of the regularities in
semantic transformations. Conversely, the extent to which word meanings
(or word relationships with the rest of the utterance) participate in
the transformation process could be explored. Better integrating these
results with what is known of the way utterances are cognitively
processed and produced could also be fruitful: known mechanisms in
sentence parsing and processing could explain the patterns observed by
our model, and integrating with current knowledge of sentence production
could make it possible to develop generative models of the
transformation process. Finally, many questions from the last chapter
also remain pending. In particular, if context is an important factor in
the transformations observed, we wonder about the role of feedback
effects and path dependence in the evolution of content online: how much
are transformations determined by the distribution of utterances that
readers are exposed to, in what way, and how does this influence feed
back into the distribution itself? An important role of feedback in the
transformations would be grounds for a strong path dependence in the
evolution of linguistic content, maybe even at lower cognitive levels.

Answering these more approachable questions would already provide a much
more complete view of the dynamics at work in the short-term evolution
of linguistic content. As noted above however, it is likely that further
progress in this area will require considering the role of pragmatics in
utterance transformations, for instance by exploring more ecological
interaction situations than the simple read-and-rewrite task we used.
